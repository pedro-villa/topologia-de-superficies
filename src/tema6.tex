\chapter{Homotopía. El grupo fundamental}

\section{Equivalencia homotópica}

\defn{Camino o arco}{
    Sea ($X,\tau$) un espacio topológico e $I = [0,1] \subset \R$. Se llama \textbf{camino o arco} en $X$ a una aplicación continua $\alpha: I \to X$. Se llama $\alpha(0)$ origen y $\alpha(1)$ final. Diremos que $\alpha$ es un arco uniendo $\alpha(0)$ con $\alpha(1)$. Cuando $\alpha(0) = \alpha(1)$ diremos que $\alpha$ es un lazo.
}

\defn{Aplicaciones homotópicas}{
    Dos aplicaciones continuas $f ,g : X \to Y$ entre dos espacios topológicos son homotópicas si existe una aplicación continua $F : X \times [0,1] \to Y$ tal que $F(x,0) = f(x)$ y $F(x,1) = g(x)$, para todo $x \in X$. Se dice que $F$ es una homotopía entre $f$ y $g$, y se representa por $f \simeq g$.
}

\rmkb{
    Si $F$ es una homotopía, para cada $t \in [0,1]$, definimos $F_t : X \to Y$ por $F_t(x) = F(x,t)$. Las aplicaciones $F_t$ son continuas, $F_0 = f$ y $F_1 = g$. Si fijamos $x$, entonces
    \[
    F_x(t) = F(x,t) : I \to Y
    \]
    es un arco en $Y$ uniendo $f(x)$ con $g(x)$.
}

\clearpage

\exer{
    Si $Y\subset \R^n$ es convexo, cualquier par de aplicaciones $f,g$ continuas son homotópicas.
}

\pf{
    Consideremos la siguiente homotopía:
    \[
    F:X\times I \to Y,\quad F(x,t)=g(x)+(1-t)f(x)
    \]
    como para cada $x\in X$ se tiene $f(x),g(x)\in Y$ y el conjunto $Y$ es convexo, entonces el segmento contenido entre $f(x),g(x)$ está contenido en $Y$, luego $F$ está bien definida. Además es continua por ser composición de aplicaciones continuas. Finalmente notemos que $F(x,0)=f(x),F(x,1)=g(x)$.
}

\exer{
    Sean $X, Y$ espacios topológicos e $y_1, y_2 \in Y$. Entonces las aplicaciones constantes
    \[
    C_{y_i} : X \to Y,\quad C_{y_i}(x) = y_i,\quad i=1,2
    \]
    son homotópicas si y solo si existe un arco en $Y$ uniendo $y_1, y_2$.
}

\pf{
    Si ambas aplicaciones son homotópicas sea $F(x,t)$ la homotopía entre ellas, entonces dado $x_0\in X$ arbitrario la aplicación $F_{x_0}(t)=F(x_0,t)$ es un camino que une $y_1,y_2$ puesto que es continua y
    \[
    F_{x_0}(0)=F(x_0,0)=C_{y_1}(x_0)=y_1,\quad F_{x_0}(1)=F(x_0,1)=C_{y_2}(x_0)=y_2.
    \]

    Si por el contrario existe $\alpha:I\to Y$ un camino uniendo $y_1=\alpha(0),y_2=\alpha(1)$ entonces consideremos la aplicación dada por
    \[
    F(x,t)=tC_{y_2}+(1-t)C_{y_1}
    \]
    que es claramente continua y además verifica $F(x,0)=C_{y_1}, F(x,1)=C_{y_2}$, luego es una homotopía.
}

\propp{Relación de equivalencia}{
    La propiedad de ser homotópicas $\simeq$ es una relación de equivalencia en el conjunto $\mathcal{C}(X,Y)$ formado por todas las aplicaciones continuas de $X$ en $Y$. 
}{
    \textbf{1. Reflexiva:} $f \simeq f$

    Definamos la aplicación $F:X\times I\to Y$ por $F(x,t) = f(x)$. $F$ es continua por ser $f$ continua, además se tiene
    $\begin{cases}
        F(x,0) = f(x) \\
        F(x,1) = f(x)
    \end{cases}$
    por lo que es una homotopía, luego $f\simeq f$.

    \clearpage

    \noindent\textbf{2. Simétrica:} $f \simeq g \implies g \simeq f$
    
    Sea $F$ la homotopía entre $f,g$ entonces la aplicación
    \[
        \overline{F}:X\times I \to Y,\quad 
        \overline{F}(x,t) = F(x,1-t)
    \]
    es una homotopía entre $g,f$ puesto que es continua al serlo $F$ y cumples $\overline{F}(x,0)=F(x,1)=g(x), \overline{F}(x,1)=F(x,0)=f(x)$. 
    
    \textit{Nota:} La aplicación $\overline{F}$ se puede construir con cualquier homeomorfismo $\alpha : I \to I$ con $alpha(0) = 1$ y $\alpha(1) = 0$ en vez de $1-t$.

    \noindent\textbf{3. Transitiva:} $f \simeq g,g \simeq h \implies f \simeq h$
    
    Sea $F_1$ la homotopía entre $f,g$ y $F_2$ la homotopía entre $g,h$, entonces la aplicación
    \[
    \overline{F}:X\times I \to Y,\quad
    \overline{F}(x,t) =\begin{cases}
        F_1(x,2t) & t \in [0, \frac{1}{2}] \\
        F_2(x,2t-1) & t \in [\frac{1}{2}, 1]
    \end{cases}
    \]
    es continua y verifica $\overline{F}(x,0)=F_1(x,0)=f(x), \overline{F}(x,1)=F_2(x,1)=h(x)$, por lo que es una homotopía entre $f$ y $h$.
    
    \textit{Nota:} Al igual que antes, se podría hacer con cualquier homeomorfismo que cumpla con lo que necesitamos.
}

\defn{Espacios homotópicamente equivalentes}{
    Decimos que dos espacios topológicos $X$ e $Y$ son homotópicamente equivalentes o tienen el mismo tipo de homotopía (denotado $X \simeq Y$) si existen aplicaciones continuas $f : X \to Y$ y $g :Y \to X$ tales que $g \circ f \simeq Id_X$ y $f \circ g \simeq Id_Y$. A la aplicación $f$ (también a $g$) se le llama, equivalencia de homotopía.
}

\exer{
    Si $Y \subset \R^n$ es convexo, entonces $Y \simeq \{x\}$.
}

\pf{
    Sea $X=\{x\}$, fijemos un $y_0\in Y$ y definamos las aplicaciones $f : X \to Y,\quad f(x)=y_0$ y $g : Y \to X,\quad g(y)=x$. Es inmediato notar que $g\circ f=Id_X$ puesto que para el único punto de $X$ se cumple
    \[
    g(f(x))=g(y_0)=x.
    \]
    Por tanto $g\circ f=Id_X\simeq Id_X$ por la reflexividad de $\simeq$.
    
    Por otro lado sea $h=f\circ g:Y\to Y$ y definamos la siguiente homotopía:
    \[
    H(y,t)= (y,t) = ty + (1-t)y_0 \implies
    \begin{cases}
        H(y,0) = y_0 = f(x) = f(g(y)) = h(y) \\
        H(y,1) = y = Id_Y(y)
    \end{cases}
    \]
    que es claramente continua y está bien definida porque $Y$ es convexo. 
}

\exer{
    $\R^2\setminus\{0\} \simeq \mathbb{S}^1$.
}

\pf{
    Buscamos $f : \R^2\setminus\{0\} \to \mathbb{S}^1$ y $g : \mathbb{S}^1 \to \R^2\setminus\{0\}$ continuas tales que
    \[
    \begin{cases}
        f \circ g \simeq Id_{\mathbb{S}^1} \\
        g \circ f \simeq Id_{\R^2\setminus\{0\}}.
    \end{cases}
    \]
    
    Sean $f(y) = \frac{y}{|y|}$ y $g(x) = x$. Es inmediato notar que $f\circ g= Id_{\mathbb{S}^1} \simeq Id_{\mathbb{S}^1}$ ya que \[
    f(g(x))=f(x)=\frac{x}{|x|}=x
    \]
    al ser $g(x)=x$ de norma 1.
    
    Por otro lado definamos la homotopía $F$ siguiente
    \[
        F(y,t)=ty + (1-t)\frac{y}{|y|},\quad
        \begin{cases}
            F(y,0) = \frac{y}{|y|}=g(f(y)) \\
            F(y,1) = y = Id_{\R^2\setminus\{0\}}(y)
        \end{cases}
    \]
    luego $g\circ f\simeq Id_{\R^2\setminus\{0\}}$.
}

\ex{
    La corona circular $C = \{(x,y) \in \R^2 : 1 \le x^2 +y^2 \le 2\} \simeq \mathbb{S}^1$
}

\propp{Espacios homotópicos y homeomorfos}{
    Si $X$ e $Y$ son homeomorfos, entonces $X \simeq Y$. El recíproco no es cierto.
}{
    Supongamos que $X$ e $Y$ son homeomorfos. En tal caso existe $f:X \to Y$ homeomorfismo, que será una aplicación continua y con inversa $g=f^{-1}$ también continua. Entonces
    \[
    g\circ f=Id_{X}\simeq Id_X,\quad f\circ g=Id_{Y}\simeq Id_{Y}
    \]
    por lo que $X\simeq Y$.

    Como contraejemplo para el recíproco consideremos el conjunto $\R^n$ que es convexo, por tanto $\R^n\simeq \{x\}$ para cualquier $x\in\R^n$, sin embargo no son espacios homeomorfos puesto que $\R^n$ no es compacto y $\{x\}$ sí.
}
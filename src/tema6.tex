\chapter{Homotopía. El grupo fundamental}

\section{Equivalencia homotópica}

\defn{Camino o arco}{
    Sea ($X,\tau$) un espacio topológico e $I = [0,1] \subset \R$. Se llama \textbf{camino o arco} en $X$ a una aplicación continua $\alpha: I \to X$. Se llama $\alpha(0)$ origen y $\alpha(1)$ final. Diremos que $\alpha$ es un arco uniendo $\alpha(0)$ con $\alpha(1)$. Cuando $\alpha(0) = \alpha(1)$ diremos que $\alpha$ es un lazo.
}

\defn{Aplicaciones homotópicas}{
    Dos aplicaciones continuas $f ,g : X \to Y$ entre dos espacios topológicos son homotópicas si existe una aplicación continua $F : X \times [0,1] \to Y$ tal que $F(x,0) = f(x)$ y $F(x,1) = g(x)$, para todo $x \in X$. Se dice que $F$ es una homotopía entre $f$ y $g$, y se representa por $f \simeq g$.
}

\rmkb{
    Si $F$ es una homotopía, para cada $t \in [0,1]$, definimos $F_t : X \to Y$ por $F_t(x) = F(x,t)$. Las aplicaciones $F_t$ son continuas, $F_0 = f$ y $F_1 = g$. Si fijamos $x$, entonces
    \[
    F_x(t) = F(x,t) : I \to Y
    \]
    es un arco en $Y$ uniendo $f(x)$ con $g(x)$.
}

\clearpage

\exer{
    Si $Y\subset \R^n$ es convexo, cualquier par de aplicaciones $f,g$ continuas son homotópicas.
}

\pf{
    Consideremos la siguiente homotopía:
    \[
    F:X\times I \to Y,\quad F(x,t)=g(x)+(1-t)f(x)
    \]
    como para cada $x\in X$ se tiene $f(x),g(x)\in Y$ y el conjunto $Y$ es convexo, entonces el segmento contenido entre $f(x),g(x)$ está contenido en $Y$, luego $F$ está bien definida. Además es continua por ser composición de aplicaciones continuas. Finalmente notemos que $F(x,0)=f(x),F(x,1)=g(x)$.
}

\exer{
    Sean $X, Y$ espacios topológicos e $y_1, y_2 \in Y$. Entonces las aplicaciones constantes
    \[
    C_{y_i} : X \to Y,\quad C_{y_i}(x) = y_i,\quad i=1,2
    \]
    son homotópicas si y solo si existe un arco en $Y$ uniendo $y_1, y_2$.
}

\pf{
    Si ambas aplicaciones son homotópicas sea $F(x,t)$ la homotopía entre ellas, entonces dado $x_0\in X$ arbitrario la aplicación $F_{x_0}(t)=F(x_0,t)$ es un camino que une $y_1,y_2$ puesto que es continua y
    \[
    F_{x_0}(0)=F(x_0,0)=C_{y_1}(x_0)=y_1,\quad F_{x_0}(1)=F(x_0,1)=C_{y_2}(x_0)=y_2.
    \]

    Si por el contrario existe $\alpha:I\to Y$ un camino uniendo $y_1=\alpha(0),y_2=\alpha(1)$ entonces consideremos la aplicación dada por
    \[
    F(x,t)=tC_{y_2}+(1-t)C_{y_1}
    \]
    que es claramente continua y además verifica $F(x,0)=C_{y_1}, F(x,1)=C_{y_2}$, luego es una homotopía.
}

\propp{Relación de equivalencia}{
    La propiedad de ser homotópicas $\simeq$ es una relación de equivalencia en el conjunto $\mathcal{C}(X,Y)$ formado por todas las aplicaciones continuas de $X$ en $Y$. 
}{
    \textbf{1. Reflexiva:} $f \simeq f$

    Definamos la aplicación $F:X\times I\to Y$ por $F(x,t) = f(x)$. $F$ es continua por ser $f$ continua, además se tiene
    $\begin{cases}
        F(x,0) = f(x) \\
        F(x,1) = f(x)
    \end{cases}$
    por lo que es una homotopía, luego $f\simeq f$.

    \clearpage

    \noindent\textbf{2. Simétrica:} $f \simeq g \implies g \simeq f$
    
    Sea $F$ la homotopía entre $f,g$ entonces la aplicación
    \[
        \overline{F}:X\times I \to Y,\quad 
        \overline{F}(x,t) = F(x,1-t)
    \]
    es una homotopía entre $g,f$ puesto que es continua al serlo $F$ y cumples $\overline{F}(x,0)=F(x,1)=g(x), \overline{F}(x,1)=F(x,0)=f(x)$. 
    
    \textit{Nota:} La aplicación $\overline{F}$ se puede construir con cualquier homeomorfismo $\alpha : I \to I$ con $alpha(0) = 1$ y $\alpha(1) = 0$ en vez de $1-t$.

    \noindent\textbf{3. Transitiva:} $f \simeq g,g \simeq h \implies f \simeq h$
    
    Sea $F_1$ la homotopía entre $f,g$ y $F_2$ la homotopía entre $g,h$, entonces la aplicación
    \[
    \overline{F}:X\times I \to Y,\quad
    \overline{F}(x,t) =\begin{cases}
        F_1(x,2t) & t \in [0, \frac{1}{2}] \\
        F_2(x,2t-1) & t \in [\frac{1}{2}, 1]
    \end{cases}
    \]
    es continua y verifica $\overline{F}(x,0)=F_1(x,0)=f(x), \overline{F}(x,1)=F_2(x,1)=h(x)$, por lo que es una homotopía entre $f$ y $h$.
    
    \textit{Nota:} Al igual que antes, se podría hacer con cualquier homeomorfismo que cumpla con lo que necesitamos.
}

\defn{Espacios homotópicamente equivalentes}{
    Decimos que dos espacios topológicos $X$ e $Y$ son homotópicamente equivalentes o tienen el mismo tipo de homotopía (denotado $X \simeq Y$) si existen aplicaciones continuas $f : X \to Y$ y $g :Y \to X$ tales que $g \circ f \simeq Id_X$ y $f \circ g \simeq Id_Y$. A la aplicación $f$ (también a $g$) se le llama, equivalencia de homotopía.
}

\exer{
    Si $Y \subset \R^n$ es convexo, entonces $Y \simeq \{x\}$.
}

\pf{
    Sea $X=\{x\}$, fijemos un $y_0\in Y$ y definamos las aplicaciones $f : X \to Y,\quad f(x)=y_0$ y $g : Y \to X,\quad g(y)=x$. Es inmediato notar que $g\circ f=Id_X$ puesto que para el único punto de $X$ se cumple
    \[
    g(f(x))=g(y_0)=x.
    \]
    Por tanto $g\circ f=Id_X\simeq Id_X$ por la reflexividad de $\simeq$.
    
    Por otro lado sea $h=f\circ g:Y\to Y$ y definamos la siguiente homotopía:
    \[
    H(y,t)= (y,t) = ty + (1-t)y_0 \implies
    \begin{cases}
        H(y,0) = y_0 = f(x) = f(g(y)) = h(y) \\
        H(y,1) = y = Id_Y(y)
    \end{cases}
    \]
    que es claramente continua y está bien definida porque $Y$ es convexo. 
}

\exer{
    $\R^2\setminus\{0\} \simeq \mathbb{S}^1$.
}

\pf{
    Buscamos $f : \R^2\setminus\{0\} \to \mathbb{S}^1$ y $g : \mathbb{S}^1 \to \R^2\setminus\{0\}$ continuas tales que
    \[
    \begin{cases}
        f \circ g \simeq Id_{\mathbb{S}^1} \\
        g \circ f \simeq Id_{\R^2\setminus\{0\}}.
    \end{cases}
    \]
    
    Sean $f(y) = \frac{y}{|y|}$ y $g(x) = x$. Es inmediato notar que $f\circ g= Id_{\mathbb{S}^1} \simeq Id_{\mathbb{S}^1}$ ya que \[
    f(g(x))=f(x)=\frac{x}{|x|}=x
    \]
    al ser $g(x)=x$ de norma 1.
    
    Por otro lado definamos la homotopía $F$ siguiente
    \[
        F(y,t)=ty + (1-t)\frac{y}{|y|},\quad
        \begin{cases}
            F(y,0) = \frac{y}{|y|}=g(f(y)) \\
            F(y,1) = y = Id_{\R^2\setminus\{0\}}(y)
        \end{cases}
    \]
    luego $g\circ f\simeq Id_{\R^2\setminus\{0\}}$.
}

\ex{
    La corona circular $C = \{(x,y) \in \R^2 : 1 \le x^2 +y^2 \le 2\} \simeq \mathbb{S}^1$
}

\propp{Espacios homotópicos y homeomorfos}{
    Si $X$ e $Y$ son homeomorfos, entonces $X \simeq Y$. El recíproco no es cierto.
}{
    Supongamos que $X$ e $Y$ son homeomorfos. En tal caso existe $f:X \to Y$ homeomorfismo, que será una aplicación continua y con inversa $g=f^{-1}$ también continua. Entonces
    \[
    g\circ f=Id_{X}\simeq Id_X,\quad f\circ g=Id_{Y}\simeq Id_{Y}
    \]
    por lo que $X\simeq Y$.

    Como contraejemplo para el recíproco consideremos el conjunto $\R^n$ que es convexo, por tanto $\R^n\simeq \{x\}$ para cualquier $x\in\R^n$, sin embargo no son espacios homeomorfos puesto que $\R^n$ no es compacto y $\{x\}$ sí.
}

\lemp{Relación de equivalencia entre espacios topológicos}{
    La propiedad $\simeq$ define una relación de equivalencia entre los espacios topológicos.
}{
    \noindent\textbf{Reflexiva:} $X \simeq X$, $f = g = Id_X$.

    \noindent\textbf{Simetrica:} $X \simeq Y$ ($\exists f:X\to Y, g:Y\to X$) $\implies Y \simeq X$ ($\bar{f}=g, \bar{g}=f$).

    \noindent\textbf{Transitiva:} $X \simeq Y$ ($\exists f_1:X\to Y, g_1:Y\to X$) y $Y \simeq Z$ ($\exists f_2:Y\to Z, g_2:Z\to Y$) 

    ¿$f:X\to Z, g:Z\to Y$? $f=f_2\circ f_1, g=g_1\circ g_2$.

    $f\circ g = f_2\circ f_1\circ g_1\circ g_2 \simeq f_2\circ Id_Y \circ g_2 = f_2\circ g_2 \simeq Id_Z$.

    $g\circ f =g_1\circ g_2\circ f_2\circ f_1 \simeq g_1\circ Id_Y\circ f_1 = g_1\circ f_1 \simeq Id_X$.
}

\section{Espacios contractiles}

\defn{Espacio contráctil}{
    Un espacio topológico X se dice que es contráctil si es homotópicamente equivalente a un punto.
}

\propp{Caracterizaciones de espacios contráctiles}{
    Sea $Y$ un espacio topológico, las siguientes afirmaciones son equivalentes:
    \begin{enumerate}
        \item $Y$ es contráctil.
        \item Existe $y_0 \in Y$ tal que $Id_Y$ y $C_{y_0}:Y\to \{y_0\}$ son homotópicas.
        \item Si $X$ es otro espacio topológico, cualquier par de aplicaciones continuas $f,g : X \to Y$ son homotópicas.
        \item Para todo $y_0\in Y$, se tiene que $C_{y_0}$ es homotópica a la $Id_Y$.
        \item Si $X$ es otro espacio topológico y $f:Y\to X$ continua, entonces existe $x_0\in X$ tal que $f \simeq C_{x_0}$. 
    \end{enumerate}
}{
    \noindent ($1 \implies 2$) Existen $Y \longleftrightarrow \{p\}$, donde $C_p:Y\to \{p\}$ es tal que $C_p(y)=p$ y $f:\{p\}\to Y$. Cumplen que $C_p\circ f \simeq Id_{\{p\}}$ y $f\circ C_p \simeq Id_Y$.

    \noindent Como $f\circ C_p \to f(C_p(y))=f(p)=:y_0\in Y$, entonces $f\circ C_p = C_{y_0}:Y\to Y$ y se tiene que $C_{y_0} \simeq Id_Y$.

    \noindent ($2 \implies 3$) Sabemos que $\exists y_0\in Y$ tal que $Id_Y \simeq C_{y_0}$. Sea $X$ espacio topológico, y $f,g:X\to Y$ continuas. Veamos que $f \simeq g$.
    $$ f = Id_Y\circ f \simeq C_{y_0}\circ f = \overline{C_{y_0}} :X \to Y \text{ tal que } \overline{C_{y_0}}(x) = y_0 $$
    $$ g = Id_Y\circ g \simeq C_{y_0}\circ g = \overline{C_{y_0}} :X \to Y \text{ tal que } \overline{C_{y_0}}(x) = y_0 $$

    \noindent Por tanto, $f \simeq g$.

    \noindent ($3 \implies 4$) Cogemos $Y=X$, $f=Id_Y, g=C_{y_0}$, con $y_0\in Y$. Aplicando la hipótesis a estos espacios y funciones se obtiene que $f\simeq g \implies Id_Y \simeq C_{y_0}, y_0\in Y$.

    \noindent ($4 \implies 5$) Sabemos que $C_{y_0} \simeq Id_Y, \forall y_0\in Y$. Sea $X$ espacio topológico y $f:Y\to X$ continua. Entonces, $f = f\circ Id_Y \simeq f\circ C_{y_0} : Y \to X$ tal que $f\circ C_{y_0}(y) = f(y_0)$. Luego, $f\circ C_{y_0} = C_{f(y_0)}:Y\to X$. Tomando $x_0 = f(y_0)$, se tiene que $f \simeq C_{x_0}$. 

    \noindent ($5 \implies 2$) Aplicamos la hipótesis al espacio $X=Y$ y la función $f=Id_Y$. Entonces, obtenemos $y_0\in Y$ tal que $f\simeq C_{y_0}$. Por tanto, $Id_Y \simeq C_{y_0}$.

    \noindent ($2 \implies 1$) Por la hipótesis, existe $y_0\in Y$ tal que $Id_Y \simeq C_{y_0}$. Las funciones buscadas son $\begin{cases}
        f = C_{y_0}: Y\to \{y_0\} \\
        g = \iota : \{y_0\} \to Y
    \end{cases}$, que cumplen que:
    $$ f\circ g = C_{y_0} \circ \iota = \overline{C_{y_0}} : \{y_0\} \to \{y_0\} = Id_{\{y_0\}} $$
    $$ g\circ f = \iota \circ C_{y_0} = \tilde{C}_{y_0} : Y \to Y \simeq Id_Y$$
}

\corp{
    Todo espacio contráctil es arcoconexo y por tanto conexo.
}{
    Gracias a la proposición anterior, aplicamos el apartado $3$ a $f,g:Y\to Y$ continua, con $C_{y_1}=f(y) = y_1\in Y, C_{y_2}=g(y)=y_2\in Y$. Entonces
    $$ C_{y_1} \simeq C_{y_2} \iff \exists F:Y\times I \to Y \text{ continua tal que } \begin{cases}
        F(y,0) = C_{y_1}(y) = y_1 \\
        F(y,1) = C_{y_2}(y) = y_2
    \end{cases}$$
    $\alpha:I \to Y, \alpha(t) = F(y_0,t)$, con $y_0\in Y$ fijo. Este camino une $y_1$ y $y_2$, para cualesquiera $y_1,y_2\in Y$. Por tanto, $Y$ es arcoconexo. 
}

\defn{Invariante homotópico}{
    Un invariante homotópico es una propiedad topológica $\partes$ que se conserva por equivalencias homotópicas, es decir, si $X \simeq Y$, entonces $X$ satisface $\partes$ si, y solo si, $Y$ satisface $\partes$.
}

\ex{
    \begin{enumerate}
        \item La conexión es un invariante homotópico.
        \item La conexión por caminos es un invariante homotópico.
        \item La compacidad NO es un invariante homotópico.
    \end{enumerate}
}

\section{Homotopía por caminos}

\defn{Caminos homotópicos}{
    Dos caminos $\alpha,\beta : I \to X$ en un espacio topológico $X$ uniendo dos puntos $x_0$ y $x_1$ se dice que son homotópicos (por caminos) si existe una homotopía $F : I \times I \to X$ entre $\alpha$ y $\beta$ tal que $F(0,t) = x_0$, $F(1,t) = x_1$, para todo $t \in I$. Lo denotaremos por $\alpha \simeq_p \beta$.
}
% TODO: Hacer dibujo

\rmkb{
    Para cada $t\in I$, $F_t$ es un camino uniendo $x_0$ y $x_1$, con $F_0 = \alpha$ y $F_1 = \beta$.
}

\propp{Relación de equivalencia $\simeq_p$}{
    $\simeq_p$ define una relación de equivalencia sobr el conjunto de caminos que unen $x_0,x_1$. Denotamos por $[\alpha]$ la clase asociada a un camino $\alpha$.
}{
    Pendiente...
}

\defn{Producto de caminos}{
    Si $\alpha$ es un camino uniendo $x_0$ y $x_1$, y $\beta$ es un camino uniendo $x_1$ y $x_2$, definimos el camino producto $\alpha * \beta$ por 
    $$\alpha * \beta(s) = \begin{cases}
        \alpha(2s) & s \in [0, \frac{1}{2}] \\
        \beta(2s - 1) & s \in [\frac{1}{2}, 1]
    \end{cases}$$
}

\prop{Propiedades}{
    \begin{enumerate}
        \item $\alpha * (\beta * \gamma) \simeq_p (\alpha * \beta) * \gamma$.
        \item Si $\epsilon_{x_0}$ es el arco constante $x_0$ e igual $\epsilon_{x_1}$, entonces
        $$\epsilon_{x_0} * \alpha \simeq_p \alpha \text{ y } \alpha * \epsilon_{x_1} \simeq_p \alpha$$
        \item Sea $\bar{\alpha} = \alpha(1-t), t\in I$. Entonces, $$\alpha * \bar{\alpha} \simeq_p \epsilon_{x_0} \text{ y } \bar{\alpha} * \alpha \simeq_p \epsilon_{x_1}$$
    \end{enumerate}
}

\defn{Conjunto de arcos}{
    Dado $X$ un espacio topológico, denotaremos por $$\Omega_{x_0}^{x_1}(X) := \{\alpha:I\to X \text{ continua con } \alpha(0) = x_0, \alpha(1)=x_1\}$$ y $$\Omega_{x_0} = \Omega_{x_0}^{x_0}(X)$$
    Llamaremos $$\pi_1(X,x_0) : \faktor{\Omega_{x_0}}{\simeq_p}$$
}

\defn{Producto de clases de homotopía}{
    El producto $*$ induce un producto bien definido para clases de homotopía:
    $$[\alpha]*[\beta]:=[\alpha*\beta]$$
}

\thm{El grupo fundamental}{
    El conjunto $\pi_1(X,x_0)$ junto con la operación producto $*$ es un grupo, denominado el \textbf{grupo fundamental} de $X$ en $x_0$.
}
\lemp{Lema previo}{
    Sean $\alpha\in\Omega_{x_0}^{x_1}(X), \beta\in\Omega_{x_1}^{x_2}(X)$. Entonces, $\alpha*\beta \simeq_p \gamma(t) = \begin{cases}
        \alpha(\frac{t}{r}) & t \in [0, r] \\
        \beta(\frac{t-r}{1-r}) & t \in [r, 1]
    \end{cases}$
}{
    Quiero ver que $\alpha*\beta$ y $\gamma$ son homotópicos por caminos, luego necesito una homotopía $H(x,t):I\times I \to X$  tal que $H(x,0) = \alpha*\beta(x)$ y $H(x,1) = \gamma(x)$, además de que $H(0,t) = x_0$ y $H(1,t) = x_2$.

    $H(x,t) = \begin{cases}
        \alpha\Big(\frac{x}{(1-t)\frac{1}{2}+tr}\Big) & x \in [0, (1-t)\frac{1}{2}+tr] \\
        \beta\Big(\frac{x-(1-t)\frac{1}{2}-tr}{1-(1-t)\frac{1}{2}-tr}\Big) & x \in [(1-t)\frac{1}{2}+tr, 1]
    \end{cases}$
    \begin{itemize}
        \item $H(x,0) = \alpha*\beta(x). \checkmark$
        \item $H(x,1) = \gamma(x). \checkmark$
        \item $H(0,t) = \alpha(0) = x_0. \checkmark$
        \item $H(1,t) = \beta(1) = x_2. \checkmark$
    \end{itemize}
}
\pf{
    Pendiente...
}

\ex{
    Si $X\subset \R^n$ convexo, entonces $\pi_1(X,x_0) = 0$ (el grupo trivial) para todo $x_0\in X$. En particular $\pi_1(\R^n,x_0) = 0$ y $\pi_1(B^n, x_0) = 0$, donde $B^n$ es la bola unidad.
}
\ex{
    $\pi_1(\R,x) = (0,+)$ (el grupo trivial se denota así).
}

\defn{}{
    Sea $\sigma$ un camino uniendo dos puntos $x_0, x_1 \in X$. Definimos
    \[
    \hat{\sigma} : \pi_1(X, x_0) \to \pi_1(X, x_1)
    \]
    por
    \[
    \hat{\sigma}([\gamma]) = [\overline{\sigma}] * [\gamma] * [\sigma].
    \]
}

\propp{Proposición 6.8}{\label{prop:iso-grupos}
    La aplicación $\hat{\sigma}$ es un isomorfismo de grupos para todo camino $\sigma$. 
}{
    Para ver que $\hat{\sigma}$ es biyectiva comprobaremos que tiene inversa dada por
    \[
    (\hat{\sigma})^{-1}=\widehat{(\overline{\sigma})}
    \]
    en efecto se tiene
    \begin{align*}
        \widehat{(\overline{\sigma})}(\hat{\sigma})([\gamma])=\text{Pasos pendientes}=[\gamma]
    \end{align*}
}

\corp{
    Si $X$ es conexo por caminos, todos los grupos $\pi_1(X, x_0)$ son isomorfos entre sí. En este caso, podemos considerar el grupo fundamental $\pi_1(X)$, independiente del punto base.
}{
    Pendiente.
}

\section{Espacios simplemente conexos}

\defn{Espacio simplemente conexo}{
    Decimos que un espacio topológico $X$ es simplemente conexo si es conexo por caminos y $\pi_1(X) = 0$.
}

\ex{
    Si $X \subset \mathbb{R}^n$ es convexo, entonces $X$ es simplemente conexo.
}

\section{El homomorfismo inducido}

\defn{Homomorfismo inducido}{
    Sea $f : X \to Y$ una aplicación continua. Para todo punto $x_0 \in X$, $f$ induce una aplicación
    \[
    f_* : \pi_1(X, x_0) \to \pi_1(Y, f(x_0))
    \]
    definida por $f_*([\alpha]) = [f \circ \alpha]$, denominada el homomorfismo inducido.
}

\propp{Propiedades del homomorfismo inducido}{
    \begin{enumerate}
        \item Si $f: X \to Y$ y $g: Y \to Z$ son continuas, entonces $(g \circ f)_* = g_* \circ f_*$.
        \item Para la identidad $Id_X: X \to X$ se tiene $(Id_X)_* = Id|_{\pi_1(X,x_0)}.$
        \item Si $f$ es un homeomorfismo, entonces $f_*$ es un isomorfismo de grupos. Así, el grupo fundamental es un invariante topológico.
    \end{enumerate}
}{
    Pendiente.
}
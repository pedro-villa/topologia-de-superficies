\appendix
\chapter{Teoría de grupos}

\section{Grupos y subgrupos}

\defn{Operación binaria}{
    Dado un conjunto $X$, una operación binaria en $X$ es una aplicación
    \[
        * : X \times X \to X
    \]
    Dados $a,b\in X$ se suele abreviar la notación escribiendo $ *(a, b) =: a * b$. 
}

\ex{
    \begin{itemize}
        \item La suma $+$ es una operación binaria sobre $\mathbb{Z}$.
        \item La resta $-$ NO es una operación binaria sobre $\mathbb{N}$.
        \item Dado un conjunto cualquiera $X$, llamemos $\Phi(X) = \{ \varphi : X \to X : \text{ } \varphi \text{ es biyectiva }\}$. La composición $\circ$ es una operación binaria sobre $\Phi(X)$.
    \end{itemize}
}

\defn{Grupo}{
    Un grupo es un par $(G, *)$ donde $G$ es un conjunto y $*: G \times G \to G$ es una operación binaria que satisface:
    \begin{enumerate}
        \item[(G1)] \textbf{Asociatividad}: $(a * b) * c = a * (b * c)$ para todo $a, b, c \in G$
        \item[(G2)] \textbf{Elemento neutro}: Existe $e \in G$ tal que $e * a = a * e = a$ para todo $a \in G$
        \item[(G3)] \textbf{Elemento inverso}: Para cada $a \in G$, existe $a^{-1} \in G$ tal que
        \[
        a * a^{-1} = a^{-1} * a = e.
        \]
    \end{enumerate}
}

\defn{Grupo abeliano}{
    Un grupo $(G, *)$ se dice abeliano o conmutativo si además cumple:
    \begin{enumerate}
        \item[(G4)] \textbf{Conmutatividad}: $a * b = b * a$ para todo $a, b \in G$
    \end{enumerate}
}

\rmkb{
    Sobre la notación usual para grupos. Si $(G, *)$ es un grupo arbitrario, por comodidad se suele trabajar con la siguiente notación: 
    \begin{itemize}
    \item Si sabemos que $G$ es abeliano, se suele usar notación sumativa. Es decir, se sustituye el símbolo $*$ por $+$ para denotar la operación, y trabajamos con $(G, +)$. Además, el elemento neutro $e_G \in G$ se suele llamar el ``cero'' de $G$, y se escribe $0_G$ o directamente $0$. Es bastante intuitivo pensar que para cualquier $g\in G$, $0_G + g = g + 0_G = g$, aunque no estemos trabajando con números. 
    \item Alternativamente y de forma más general, se puede usar notación multiplicativa: trabajamos con $(G, \cdot)$ y ahora el neutro es $e_G = 1_G$ el ``uno'' de $G$. Se tiene que $\forall g \in G$, $1_G \cdot g = g \cdot 1_G = g$. De nuevo, recalco, no estamos trabajando con números.
    \end{itemize}
}

\exer{
    Probar las siguientes afirmaciones:
    \begin{itemize}
        \item $(\mathbb{Z}, +)$ es un grupo abeliano.
        \item $(\R\setminus\{0\}, \cdot)$ es un grupo abeliano.
        \item El grupo simétrico $S_n$ de permutaciones de $n$ elementos es un grupo con la composición como operación. 
        \item $(\N, +)$ no es un grupo.
    \end{itemize}
}

\rmkb{
    Dado un conjunto cualquiera $X$ con $n$ elementos, una permutación en $X$ es una aplicación biyectiva $\sigma : X \to X $. Cuando se habla del grupo simétrico es lo mismo trabajar con el conjunto $X$ que con cualquier otro que tenga el mismo cardinal. Por ello, se suelen describir las permutaciones como aplicaciones $ \sigma : \mathbb{N}_n \to \mathbb{N}_n$, donde $\mathbb{N}_n = \{1, 2, \dots, n\}$.
}

\defn{Subgrupo}{
    Un subconjunto $H$ de un grupo $G$ es un subgrupo (denotado $H \leq G$) si:
    \begin{enumerate}
        \item $H$ es no vacío
        \item Para todo $a, b \in H$, $a * b \in H$ (cerrado bajo la operación)
        \item Para todo $a \in H$, $a^{-1} \in H$ (cerrado bajo inversos)
    \end{enumerate}
}
\propp{Observacion sobre subgrupos}
{Todo subgrupo es un grupo.}
{
    Si $H \leq G$, entonces $(H, *)$ es un grupo con la misma operación que $G$. 
    Ya se tienen las propiedades de la asociatividad y la la existencia del elemento inverso, solo hace falta ver que existe un elemento neutro, que además, resulta coincidir con el elemento neutro de $G$. \newline
    
    Como $H$ es no vacío, existe $a \in H$. Entonces definimos el neutro de $H$ como \newline $e_H = a * a^{-1} \in H$. Observemos que como $a\in G$, también se tiene que $a * a^{-1} = e_G$ y ambos neutros son en realidad el mismo elemento. De aquí se deduce que $e_H$ cumple las propiedades de elemento neutro al heredarlas de $G$.
}

\section{Homomorfismos}

\defn{Homomorfismo de grupos}{
    Sean $(G, \cdot)$ y $(H, *)$ grupos. Una aplicación $f: G \to H$ es un homomorfismo de grupos si preserva las operaciones de grupo, es decir:
    \[
    f(a \cdot b) = f(a) * f(b) \quad \forall a, b \in G
    \]
}

\defn{Isomorfismo de grupos}{
    Un isomorfismo de grupos es una aplicación biyectiva que además es homomorfismo. Si $(G, \cdot)$ y $(H, *)$ son grupos y $f: G \to H$ es un isomorfismo decimos que $G$ y $H$ son isomorfos. 
}

\defn{Imagen y núcleo}{
    Dado un homomorfismo $f: G \to H$ definimos la imagen de $f$ como
    \[
        \text{Im} f = \{ h \in H \mid h = f(g), g \in G\}
    \]
    y el núcleo de $f$ como 
    \[
        \text{Ker} f = \{ g \in G \mid f(g)=e_H \}
    \]
    donde $e_H$ indica el elemento neutro del grupo $H$.
}

\exer{
    Sea $f: G \to H$ un homomorfismo, probar las siguientes afirmaciones:
    \begin{itemize}
        \item $f$ es inyectiva si y solo si $\text{Ker} f=\{e_H\}$.
        \item Si $f$ es biyectiva, entonces $f^{-1}$ también es un homomorfismo.
        \item $\text{Im} f,\text{Ker} f$ son subgrupos de $H,G$ respectivamente.
        \item Si $K \leq G$ entonces $f(K) \leq H$. Recordar que $\leq$ indica que el conjunto es un subgrupo. 
    \end{itemize}
}
\appendix
\chapter{Teoría de grupos}

\section{Grupos y subgrupos}

\defn{Grupo}{
    Un grupo es un par $(G, *)$ donde $G$ es un conjunto y $*: G \times G \to G$ es una operación binaria que satisface:
    \begin{enumerate}
        \item[(G1)] \textbf{Asociatividad}: $(a * b) * c = a * (b * c)$ para todo $a, b, c \in G$
        \item[(G2)] \textbf{Elemento neutro}: Existe $e \in G$ tal que $e * a = a * e = a$ para todo $a \in G$
        \item[(G3)] \textbf{Elemento inverso}: Para cada $a \in G$, existe $a^{-1} \in G$ tal que
        \[
        a * a^{-1} = a^{-1} * a = e.
        \]
    \end{enumerate}
}

\defn{Grupo abeliano}{
    Un grupo $(G, *)$ se dice abeliano o conmutativo si además cumple:
    \begin{enumerate}
        \item[(G4)] \textbf{Conmutatividad}: $a * b = b * a$ para todo $a, b \in G$
    \end{enumerate}
}

\exer{
    Probar las siguientes afirmaciones:
    \begin{itemize}
        \item $(\mathbb{Z}, +)$ es un grupo abeliano.
        \item $(\R\setminus\{0\}, \cdot)$ es un grupo abeliano.
        \item El grupo simétrico $S_n$ de permutaciones de $n$ elementos es un grupo.
        \item $(\N, +)$ no es un grupo.
    \end{itemize}
}

\defn{Subgrupo}{
    Un subconjunto $H$ de un grupo $G$ es un subgrupo (denotado $H \leq G$) si:
    \begin{enumerate}
        \item $H$ es no vacío
        \item Para todo $a, b \in H$, $a * b \in H$ (cerrado bajo la operación)
        \item Para todo $a \in H$, $a^{-1} \in H$ (cerrado bajo inversos)
    \end{enumerate}
}

\section{Homomorfismos}

\defn{Homomorfismo de grupos}{
    Sean $(G, \cdot)$ y $(H, *)$ grupos. Una aplicación $f: G \to H$ es un homomorfismo de grupos si preserva las operaciones de grupo, es decir:
    \[
    f(a \cdot b) = f(a) * f(b) \quad \forall a, b \in G
    \]
}